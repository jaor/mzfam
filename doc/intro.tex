
\chapter{Introduction}
\label{cha:introduction}

\MzFam\ consists of a set of PLT-scheme modules providing utilities to
monitor and react to filesystem changes. It exports a high-level
interface consisting of monitoring tasks that run as independent
threads and invoke callback procedures each time a file alteration is
detected. These high-level tasks (provided by the module
\scm{fam-task}) are implemented using either Linux's
\urlh{http://oss.sgi.com/projects/fam/}{FAM}/\urlh{http://www.gnome.org/~veillard/gamin/}{Gamin}
monitors or, in systems (like Windows) where it is not available, a
pure Scheme fall-back implementation. This low-level functionality is
in turn provided by two modules (\scm{fam} and \scm{fam-mz},
respectively) that share the generic interface defined by a third one,
\scm{fam-base}. You may use \scm{fam} or \scm{fam-mz} directly
(bypassing the high-level interface) if you find \scm{fam-task} not
suitable for your needs. The generic functions in \scm{fam-base} are,
basically, one-to-one counterparts of those provided by the
\texttt{libfam} C library\numfootnote{See also
  \urlh{http://techpubs.sgi.com/library/tpl/cgi-bin/getdoc.cgi/0650/bks/SGI_Developer/books/IIDsktp_IG/sgi_html/ch08.html#LE33384-PARENT}{this
    tutorial on the FAM C library} for further, unneeded details.}.

\htmlonly
\rawhtml<img align="right" vspace="4" hspace="4" src="mzfam.png"/>\endrawhtml
\endhtmlonly

Using FAM tasks is quite straightforward. So, the impatient among you
may prefer to take a look at the procedures exported by
\texttt{fam-task.ss} (and perhaps the sample scripts in the aptly
named \texttt{examples} subdirectory) and skip the rest of this
manual. For those of you still here, chapter~\ref{cha:fam-tasks} is a
tutorial on FAM tasks, and should provide all the information
needed to use \MzFam\ in your programs. In
chapter~\ref{cha:scmf-task-refer}, you'll find reference sections
(written using the dry, legalistic tone we all know and love) with all
the details of the exported \scm{fam-task} interface.

In case you need more fine-grained control over your file-monitoring,
turn to chapter~\ref{cha:low-level-interface}, which describes the
generic interface exported by the \scm{fam-base} module and
implemented by \scm{fam} (using \texttt{libfam} (or
\texttt{libgamin})) and by \scm{fam-mz} (using Scheme).

\section{Getting MzFAM}
\label{sec:getting-mzfam}

The latest \MzFam\ stable release
\urlh{http://planet.plt-scheme.org/display.ss?package=mzfam.plt&owner=jao}{is
  available} on \urlh{http://planet.plt-scheme.org}{PLaneT\\ (\1)},
where you can also browse the source code and documentation. As with
any other PLaneT-enabled package, using a require line like this one
\scm{
(require (planet jao/mzfam/fam-task))
}
anywhere in your program will trigger the network magic necessary
to get \MzFam\ installed in your system.

If you feel like living on the bleeding edge, the development tree
(which includes build scripts and the documentation sources) is
available \urlh{http://gitorious.org/mzfam}{at gitorious\\ (\1)}.

Needless to say, you're not only welcome, but actually encouraged to
\urlh{mailto:jao@gnu.org}{send the author} bug reports, improvement
suggestions, constructive criticism or any kind of undeserved praise.


%%% Local Variables:
%%% mode: latex
%%% TeX-master: "mzfam"
%%% End:
