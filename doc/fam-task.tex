
\chapter{FAM Tasks}
\label{cha:fam-tasks}

This chapter describes \MzFam's high-level interface, which,
hopefully, will fulfil all you file monitoring needs (if that's not
the case you can either \urlh{mailto:jao@gnu.org}{bug me} to improve the
interface or roll your own using the low-level interface detailed in
subsequent chapters).

We begin with some concrete usage examples in
section~\ref{sec:basic-usage} to get the gist of how FAM tasks work
(as you'll see, there's no rocket science involved), followed by a
review of some knobs you can use to customize their behaviour in
section~\ref{sec:nuts-bolts}. Finally, section~\ref{sec:fam-events}
tells you all you need to know about what and when FAM events are
generated by the system in response to file alterations.

\section{Basic usage}
\label{sec:basic-usage}

A FAM task is just a thread that monitors a set of files or
directories. Whenever it notices a change in any of them, the task
checks whether you've expressed interest on being notified for that
kind of change and invokes the corresponding procedure (also provided
by you as a token of your interest) with a so-called \textit{FAM
  event} as its argument. A FAM event is just a struct containing the
monitored path, the path of the altered file (which may be different
from the monitored path when the latter is a directory), a symbol (or
\textit{event type}) describing the type of alteration (file modified,
created, deleted and so on) and a timestamp.

So, using FAM tasks is a simple matter of:
\begin{itemize}
\item Creating a FAM task instance using \scm{(fam-task-create)}.
\item Registering one or more paths to be monitored (using
  \scm{fam-task-add-path}), providing, for each one, a procedure
  handling events and, optionally, a list of event types you're
  interested on.
\item Starting (or joining) the FAM task using \scm{fam-task-start}
  (\scm{fam-task-join}).
\end{itemize}
Let us spell out this process in a fully functional, if a bit silly,
program:
\scm{
(require (planet "fam-task.ss" ("jao" "mzfam.plt" 1 0)))

(define ft (fam-task-create))

(define (display-event event)
  (let ((mp (fam-event-monitored-path event))
        (p (fam-event-path event))
        (type (fam-event-type->string (fam-event-type event)))
        (time (fam-event-timestamp event)))
    (printf "~A: ~A while monitoring ~A (~A)~%" p type mp time)))

(when (fam-task-add-path ft "/home/jao/tmp" display-event)
  (fam-task-join ft))

(error "Ooops, fam task exited unexpectedly!")
}
The above code shows the four selectors available to extract relevant
information from FAM events you receive, besides an otherwise useless
function that comes in handy when writing tutorial snippets. You can
also see the most simple way of adding a path to the set monitored by
a FAM task (using \scm{fam-task-add-path}). Note that the monitoring
thread is not started until we call \scm{fam-task-join}, which, in
addition, makes the current thread wait on the FAM task's. The
alternative is to use \scm{fam-task-start}, which returns after
spawning the monitoring thread and let's you go on with your business.
Those business may include, incidentally, adding new paths to be
monitored or removing previously added ones (as in
\scm{(fam-task-remove-path ft "/home/jao/tmp")}), or even stoping the
monitoring task (\scm{(fam-task-stop ft)}). This slightly more
interesting program exercises all those abilities:
\scm{
(require (planet "fam-task.ss" ("jao" "mzfam.plt" 1 0)))

(define ft (fam-task-create))

(define (display-event event)
  ;; as above ...)

(unless (fam-task-start ft)
  (error "Could not start monitoring task"))

(printf "Monitoring using ~A started~%"
        (if (fam-use-native?) "scheme FAM" "FAM/Gamin daemon"))

(define (display-event event)
  (printf "* ~A: ~A (~A)~%"
          (fam-event-path event)
          (fam-event-type->string (fam-event-type event))
          (fam-event-monitored-path event)))

(define (read-op)
  (printf "(a)dd, (r)emove, (s)uspend, r(e)sume, (p)rint, (q)uit: ")
  (let ((op (read))) (read-line) op))

(define (read-path) (printf "Path: ") (read-line))

(let loop ((op (read-op)))
  (if (case op
        ((p) (display (fam-task-monitored-paths ft)) (newline) #t)
        ((a) (fam-task-add-path ft (read-path) display-event))
        ((r) (fam-task-remove-path ft (read-path)))
        ((s) (fam-task-suspend-monitoring ft (read-path)))
        ((e) (fam-task-resume-monitoring ft (read-path)))
        (else #f))
      (display "OK")
      (display "KO"))
  (newline)
  (if (eq? op 'q) (fam-task-stop ft) (loop (read-op))))

(when (not (null? (fam-task-monitored-paths ft)))
  (display "You left some paths to monitor...")
  (unless (fam-task-join ft) (error "Couldn't restart monitoring")))
}

This example brings into play all the procedures exported by
\scm{fam-task}, with the exception of the predicte
\scm{fam-task-running?} (whose meaning I'll leave to the reader's
imagination). As you have surely noticed, there's no problem
restarting a stopped FAM task, the only caveat being that it remembers
the paths it was monitoring but will not catch any event that happened
while it was dead. One can also suspend and resume monitoring of a
previously added path, the difference with just add-remove-add being
that events associated with a suspended path are duly reported upon
resuming. The parameter \scm{fam-use-native?} tells us whether \MzFam\
has been able to contact a running FAM/Gamin daemon (and the
associated C library): if that's the case, \scm{(fam-use-native?)}
evaluates to \scm{#f}; otherwise the fall-back, pure-scheme
implementation is used.

Incidentally, the monitoring program above has the dubious virtue of
exposing a situation where you may be tempted to use one of the
low-level modules directly: you just need to play with it a little and
get nervous when the output of \scm{display-event} gets mixed with the
loop's prompts. But of course, judicious use of MzScheme thread
synchronisation primitives should be a better option.

That's basically it. The next section explores some of the available
tweaks at your disposal to fine-tune how FAM tasks work.

\section{Nuts and bolts}
\label{sec:nuts-bolts}

\subsection{FAM task flavours}
\label{sec:fam-task-flavours}

As mentioned, \scm{fam-task-create} will do its very best to contact a
running FAM or Gamin daemon and only when that attempt fails will fall
back to using the pure-Scheme monitor. Nevertheless, you can instruct
\MzFam\ to use the scheme implementation (provided by the \scm{fam-mz}
module) no matter what with \scm{(fam-use-native? #t)}, either in a
line of its own or as part of a \scm{parameterize} form. The native
implementation tries hard to provide the same functionality as the
FAM-based one, but falls short in a few (hopefully minor) respects:
\begin{itemize}
\item Some, let's say, \textit{exotic} events, such as stop/start
  execution (see section~\ref{sec:fam-events} for details) are not
  detected, because the Scheme implementation operates exclusively on
  the basis of file modification times.
\item Blocking monitoring (see below) is only supported in FAM-based
  monitoring.
\end{itemize}

As hinted above, FAM tasks can operate in two modes, namely, polling
and blocking. In polling mode (the default) the thread sleeps for a
while, checks for new events, invokes the needed callback procedures,
processes requests for new, resumed or suspended paths and goes back
to sleep. The sleep period can be provided as an optional argument to
\scm{fam-task-create} whose default value is \scm{0.01} (i.e., ten
milliseconds). Thus, if you want your FAM task to perfom its duties
(approximately) every half second, just create it with
\scm{(fam-task-create 0.5)}.

When the FAM/Gamin daemon is available, FAM tasks can operate in
blocking mode: they sleep as long as there're no events. You request
this operation mode using \scm{0} as the argument passed to
\scm{fam-task-create}. In principle, this may sound like a good idea
in order to save a few CPU cycles, but a task that awakes every, say,
10 ms is not what you'd call a resource hog and, more importantly,
your FAM task won't be able to attend new monitoring requests unless a
new event awakes it. Thus, blocking operation is probably only a good
idea when the set of monitored paths is fixed before starting the FAM
task at hand.

\subsection{Path monitoring knobs}
\label{sec:path-monit-knobs}



\section{FAM events}
\label{sec:fam-events}



%%% Local Variables:
%%% mode: latex
%%% TeX-master: "mzfam"
%%% End:
