
\chapter{FAM Tasks}
\label{cha:fam-tasks}

This chapter describes \MzFam's high-level interface, which,
hopefully, will fulfil all you file monitoring needs (if that's not
the case you can either \urlh{mailto:jao@gnu.org}{bug me} to improve the
interface or roll your own using the low-level interface detailed in
subsequent chapters).

We begin with some concrete usage examples in
section~\ref{sec:basic-usage} to get the gist of how FAM tasks work
(as you'll see, there's no rocket science involved), followed by
reference sections giving you all the details of the available APIs
(using the dry, legalistic tone we all know and love).

\section{Basic usage}
\label{sec:basic-usage}

A FAM task is just a thread that monitors a set of files or
directories. Whenever it notices a change in any of them, the task
checks whether you've expressed interest on being notified for that
kind of change and invokes the corresponding procedure (also provided
by you as a token of your interest) with a so-called \textit{FAM
  event} as its argument. A FAM event is just a struct containing the
monitored path, the path of the altered file (which may be different
from the monitored path when the latter is a directory), a symbol (or
\textit{event type}) describing the type of alteration (file modified,
created, deleted and so on) and a timestamp.

So, using FAM tasks is a simple matter of:
\begin{itemize}
\item Creating a FAM task instance using \scm{(fam-task-create)}.
\item Registering one or more paths to be monitored (using
  \scm{fam-task-add-path}), providing, for each one, a procedure
  handling events and, optionally, a list of event types you're
  interested on.
\item Starting (or joining) the FAM task using \scm{fam-task-start}
  (\scm{fam-task-join}).
\end{itemize}
Let us spell out this process in a fully functional, if a bit silly,
program:
\scm{
(require (planet "fam-task.ss" ("jao" "mzfam.plt" 1 0)))

(define ft (fam-task-create))

(define (display-event event)
  (let ((mp (fam-event-monitored-path event))
        (p (fam-event-path event))
        (type (fam-event-type->string (fam-event-type event)))
        (time (fam-event-timestamp event)))
    (printf "~A: ~A while monitoring ~A (~A)~%" p type mp time)))

(when (fam-task-add-path ft "/home/jao/tmp" display-event)
  (fam-task-join ft))

(error "Ooops, fam task exited unexpectedly!")
}
The above code shows the four selectors available to extract relevant
information from FAM events you receive, besides an otherwise useless
function that comes in handy when writing tutorial snippets. You can
also see the most simple way of adding a path to the set monitored by
a FAM task (using \scm{fam-task-add-path}). Note that the monitoring
thread is not started until we call \scm{fam-task-join}, which, in
addition, makes the current thread wait on the FAM task's. The
alternative is to use \scm{fam-task-start}, which returns after
spawning the monitoring thread and let's you go on with your business.
Those business may include, incidentally, adding new paths to be
monitored or removing previously added ones (as in
\scm{(fam-task-remove-path ft "/home/jao/tmp")}), or even stoping the
monitoring task (\scm{(fam-task-stop ft)}). This slightly more
interesting program exercises all those abilities:
\scm{
(require (planet "fam-task.ss" ("jao" "mzfam.plt" 1 0)))

(define ft (fam-task-create))

(define (display-event event)
  ;; as above ...)

(unless (fam-task-start ft)
  (error "Could not start monitoring task"))

(printf "Monitoring using ~A started~%"
        (if (fam-use-native?) "scheme FAM" "FAM/Gamin daemon"))

(define (display-event event)
  (printf "* ~A: ~A (~A)~%"
          (fam-event-path event)
          (fam-event-type->string (fam-event-type event))
          (fam-event-monitored-path event)))

(define (read-op)
  (printf "(a)dd, (r)emove, (s)uspend, r(e)sume, (p)rint, (q)uit: ")
  (let ((op (read))) (read-line) op))

(define (read-path) (printf "Path: ") (read-line))

(let loop ((op (read-op)))
  (if (case op
        ((p) (display (fam-task-monitored-paths ft)) (newline) #t)
        ((a) (fam-task-add-path ft (read-path) display-event))
        ((r) (fam-task-remove-path ft (read-path)))
        ((s) (fam-task-suspend-monitoring ft (read-path)))
        ((e) (fam-task-resume-monitoring ft (read-path)))
        (else #f))
      (display "OK")
      (display "KO"))
  (newline)
  (if (eq? op 'q) (fam-task-stop ft) (loop (read-op))))

(when (not (null? (fam-task-monitored-paths ft)))
  (display "You left some paths to monitor...")
  (unless (fam-task-join ft) (error "Couldn't restart monitoring")))
}

This example brings into play almost all the interesting procedures
exported by \scm{fam-task}. As you have surely noticed, there's no
problem restarting a stopped FAM task, the only caveat being that it
remembers the paths it was monitoring but will not catch any event
that happened when it was dead. One can also suspend and resume
monitoring of a previously added path, the difference with just
add-remove-add being that events associated with a suspended path are
duly reported upon resuming. The parameter \scm{fam-use-native?}
tells us whether \MzFam\ has been able to contact a running FAM/Gamin
daemon (and the associated C library): if that's the case,
\scm{(fam-use-native?)} evaluates to \scm{#f}; otherwise the fall-back
pure-scheme implementation is used\numfootnote{You can instruct \MzFam\
  to use the scheme implementation (provided by the \scm{fam-mz}
  module) no matter what with \scm{(fam-use-native? #t)} either in a
  line of its own or as part of a \scm{parameterize} form.}.

%%% Local Variables:
%%% mode: latex
%%% TeX-master: "mzfam"
%%% End:
