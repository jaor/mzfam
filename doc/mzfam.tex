\newcommand{\MzFam}{\textit{MzFAM}}

\usepackage{graphicx}

\documentclass[twoside]{report}

\htmlimageformat{png}

\title{MzFAM: A File Alteration Monitor for MzScheme}
\author{Jos\'e Antonio Ortega Ruiz <jao@gnu.org>}
\date{}

\tableofcontents

\chapter{Introduction}
\label{cha:introduction}

\htmlonly
\rawhtml<img align="right" vspace="4" hspace="4" src="mzfam.png"/>\endrawhtml
\endhtmlonly

\MzFam\ provides a series of Scheme utilities to monitor and react to
filesystem changes. It exports a high-level interface consisting of
monitoring tasks that run as independent threads and invoke callback
procedures each time a file alteration is detected. These high-level
tasks (provided by the module \scm{fam-task}) are implemented using
either Linux's FAM/Gamin monitor or, in systems (like Windows) where
it is not available, a pure Scheme fall-back implementation. This
low-level functionality is in turn provided by two modules (\scm{fam}
and \scm{fam-mz}, respectively) which share the generic interface
defined by the \scm{fam-base} module. You may use \scm{fam} or
\scm{fam-mz} directly (bypassing the high-level interface) if you find
\scm{fam-task} not suitable for your needs. The generic functions in
\scm{fam-base} are, basically, one-to-one counterparts of those
provided by the \texttt{libfam} C library.


Chapter~\ref{cha:fam-tasks} describes FAM tasks, and should provide
all the information needed to use \MzFam\ in your programs. In case
you need more fine-grained control over your file-monitoring, turn to
chapter~\ref{cha:low-level-interface}, which describes the generic
interface exported by the \scm{fam-base} module and implemented by
\scm{fam} (using \texttt{libfam} (or \texttt{libgamin})) and by
\scm{fam-mz} (using Scheme).


\chapter{FAM Tasks}
\label{cha:fam-tasks}


%%% Local Variables:
%%% mode: latex
%%% TeX-master: "mzfam"
%%% End:


\chapter{Low-level interface}
\label{cha:low-level-interface}

\section{Basics}
\label{sec:basics}

Although \scm{fam-task} does its very best to offer you an appealing
interface, chances are some of you will find it unsuitable or simply
dislike it. Besides telling me what you'd like to see in
\scm{fam-task}, you can use the low-level monitoring API described in
this chapter as an alternative. This API, defined as a set of generic
procedures exported by the \scm{fam-base}, is a more or less direct
translation of the C interface provided by \texttt{libfam}, and is
implemented by the \scm{fam} module using PLT's FFI and by the
\scm{fam-mz} module using pure Scheme. Each of the latter modules
provides a constructor (\scm{make-fam} and \scm{make-mz-fam}) that
returns a FAM connection object to be used as the first argument of
every generic procedure.

So, the general idea is that you pick an underlying implementation and
require it together with \scm{fam-base}, and happily proceed to use
it:
\scm{
(require "fam-base.ss" (planet "jao" "mzfam.plt" 1 0))

(require "fam.ss" (planet "jao" "mzfam.plt" 1 0)) ;; or "fam-mz.ss"

(define fc (make-fam)) ;; or (make-mz-fam)
;; ... use fc via fam-base's generic functions ...
}

In addition, \scm{fam-base} exports all bindings in Swindle, so that
your implementation can take advantage of this fine CLOS clone, if you
feel like that. That's, by the way, the reason I keep calling the
procedures exported by \scm{fam-base} \textit{generic}: they're
actually implemented as generic functions. FAM events are instances of
a Swindle class, \scm{<fam-event>}, which is also exported for your
convenience.

The next section gives you all the details you need to know in order
to use \scm{fam-base}'s interface. Note that you can also look at this
interface \textit{from below}, that is, as a provider of a new native
implementation of FAM primitives (Windows anyone?). To wear this hat,
you'll have to create you own FAM connection type and provide a
constructor and concrete method implementation for the generics below.
In exchange, you'll get FAM tasks based on your new implementation for
free.

\section{Generic low-level interface}
\label{sec:generic-low-level}

As stated in the previous section, all procedures below take as first
argument a FAM connection object created with either \scm{make-fam}
(from \scm{fam}) or \scm{make-mz-fam} (from \scm{fam-mz}).

\begin{itemize}
\item \scm{(fam-monitor-path fc path)} adds \scm{path} (a string) to
  the set of paths monitored by \scm{fc}.
\item \scm{(fam-monitored-paths fc)} returns the list of (absolute)
  paths currently monitored by \scm{fc}.
\item \scm{(fam-suspend-path-monitoring fc
    path)}/\scm{fam-resume-path-monitoring fc path} suspends (resumes)
  monitoring of the given \scm{path}. Events occurring during
  suspension are remembered.
\item \scm{(fam-cancel-path-monitoring fc path)} removes \scm{path}
  from the list of monitored paths.
\item \scm{(fam-any-event? fc)} tests whether there are pending events.
\item \scm{(fam-next-event fc [wait])} grabs the next available event
  (an instance of \scm{<fam-event>}). If the optional parameter
  \scm{wait} is \scm{#t}, the call blocks until an event is available
  (the default is a non-blocking call).
\item \scm{(fam-pending-events fc)} returns a list of all available
  events. This procedure is always non-blocking.
\end{itemize}

In addition, \scm{fam-base} exports the FAM event selectors and the
auxiliar procedure \scm{fam-event-type->string} described at the end
of section~\ref{cha:scmf-task-refer}. These procedures are actually
implemented by \scm{fam-base}, unlike the generic functions above.


%%% Local Variables:
%%% mode: latex
%%% TeX-master: "mzfam"
%%% End:



%%% Local Variables:
%%% mode: latex
%%% TeX-master: t
%%% End:
